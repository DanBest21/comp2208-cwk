\documentclass{article}
\usepackage[utf8]{inputenc}
\usepackage{fancyhdr}
\usepackage{parskip}
\usepackage[a4paper, total={6in, 8in}]{geometry}

\title{COMP2208 - Search Methods Coursework}
\author{Daniel Best (Student ID: 29777127)}

\pagestyle{fancy}
\fancyhf{}
\renewcommand{\headrulewidth}{0pt}
\renewcommand{\footrulewidth}{0.5pt}
\lfoot{Daniel Best}
\cfoot{\thepage}
\rfoot{COMP2208 Coursework}

\begin{document}
	\maketitle
	
	\newpage
	
	\section{Approach}
	My approach to this assignment was to first create support classes that would handle any details that didn't directly relate to the search algorithm itself. These are as follows:
	
	\subsection{Node}
	This class defines both a singular node in the tree structure, as well as the entire tree structure itself by means of its \textbf{parent} and \textbf{children} variables. It contains a single \textbf{Grid} object, the \textbf{value} variable. It also implements the Comparable interface, where it compares an optional \textbf{estimatedCost} variable - a feature that was specifically added for the A* Search algorithm.

	\subsection{Grid}
	A class that handles the state by means of manipulating a \textbf{char[][]} multidimensional array. This class stores the actual state of the problem, storing the location of the \textbf{agent} and all of the non-white space blocks; it also allows for that state to be manipulated in a multitude of ways:
	
	\begin{itemize}
		\item Generates the start and solution state.
		\item Moves the agent in the grid, thereby changing the location of both the agent and the block it moves to.
		\item Calculates the \textbf{Manhattan distance} between the Grid and another Grid object passed to it, which is used as the heuristic for A* Search.
	\end{itemize}
	
	\subsection{Search}
	An abstract class that defines a common start and solution state for each of the individual search methods, and provides a common \textbf{expandNode()} method, which generates the children of a given node.
	
	Using this \textbf{Search} class, I was able to easily implement a class for each of the four search algorithms:
	
	\subsection{Breadth First Search (BFS)}
	Uses a \textbf{Queue} to store expanded nodes, meaning nodes are checked in the order they are expanded.
	
	\subsection{Depth First Search (DFS)}
	Uses a \textbf{Stack} to store expanded nodes, meaning the last node to expanded is checked next.
	
	\subsection{Iterative Deepening Search (IDS)}
	Uses \textbf{Depth Limited Search (DLS)}, a modified version of \textbf{DFS} that does not expand nodes at a given depth, which then iteratively increases this limit. 
	
	\subsection{A* Heuristic Search}
	Makes use of an evaluation function to determine which node to pick next, which is the \textbf{depth} of the node plus the \textbf{Manhattan distance (heuristic)} to the solution.
	
	\section{Evidence of Search Methods}
	\subsection{Breadth First Search (BFS)}
	
	\subsection{Depth First Search (DFS)}
	
	\subsection{Iterative Deepning Search (IDS)}
	
	\subsection{A* Heuristic Search}
	
	\section{Scalability Study}
	\section{Extras and Limitations}
	\section{References}
	\section{Code}
\end{document}